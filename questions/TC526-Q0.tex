% Author: Dr. Matthias Jung, DL9MJ
% Year: 2020
\documentclass[convert = false, border=5pt]{standalone}
\usepackage{fontspec}
\setmainfont{Roboto}
\usepackage[siunitx, straightvoltages, europeanresistors, european inductor]{circuitikzgit}
\usepackage{tikz}


\usepackage[siunitx, straightvoltages]{circuitikzgit}
\usepackage{amsmath}
\usepackage{unicode-math}
\setmathfont{Fira Math}
\setmathfont[range=up]{Roboto}
\setmathfont[range=it]{Roboto-Italic}
\setmathfont[range=\int]{Fira Math}
\usepackage[euler]{textgreek}
\ctikzset{%
    quadpoles/transformer core/height=2.15
}%
\begin{document}
\begin{circuitikz}
    \draw (0.8, 0) -- ++(4.6,0.0) node[inputarrow]{};
    \draw (1.0,-2) -- ++(0.0,4.5) node[inputarrow, rotate=90]{};
    \draw[ultra thick, rounded corners=0.2]
        (1,0) sin ++(0.5, 1) cos ++(0.5,-1)
              sin ++(0.5,-1) cos ++(0.5, 1)
              sin ++(0.5, 2) cos ++(0.5,-2)
              sin ++(0.5,-2) cos ++(0.5, 2);
    \draw (5.6,0) node[]{t};
    \draw[dashed] (0.8, 2) -- ++(4.6,0.0);
    \draw (0.4,2) node[]{1\,V};
\end{circuitikz}
\begin{circuitikz}
    %\tikzstyle{help lines}=[blue!50];
    %\draw[style=help lines] (-6,-4) grid (16,4);
    \draw(0,0) to [short] ++(8,0);
    \draw(0,3) to [R] ++(4,0) 
               node[circ](dc1){}
               to [short] ++(2,0)
               node[circ](dc2){}
               to [short] ++(2,0);
    \draw(4,0) node[circ]{} to [stroke diode] (dc1);
    \draw(dc2) to [stroke diode] ++(0,-3) node[circ]{};
    \draw(0,3) node[ocirc] (U1A) {};
    \draw(0,0) node[ocirc] (U1B) {};
    \draw(8,3) node[ocirc] (U2A) {};
    \draw(8,0) node[ocirc] (U2B) {};

    \draw(U1A) to[open, v=$\mbox{U}_1$] (U1B);
    \draw(U2A) to[open, v^=$\mbox{U}_2$] (U2B);
\end{circuitikz}
\end{document}

