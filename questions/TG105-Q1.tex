% Author: Dr. Matthias Jung, DL9MJ
% Year: 2020
\documentclass[convert = false, border=5pt]{standalone}
\usepackage{fontspec}
\setmainfont{Roboto}
\usepackage[siunitx, straightvoltages, europeanresistors, european inductor]{circuitikzgit}
\usepackage{tikz}



\begin{document}

\begin{circuitikz}
    \draw[line width=0.8pt] (2.5,1.5) rectangle ++ (2,1);
    \draw (3.5,2.25) node[](){Oszillator};
    \draw (3.5,1.75) node[](){47,9\,MHz};
    \draw (4.5,2) -- ++(1,0) to node[inputarrow]{} ++(0,0);
    %
    \draw[line width=0.8pt] (5.5,1.5) rectangle ++ (2,1);
    \draw (6.5,2) node[](){A};
    \draw (7.5,2) -- ++(1,0) to node[inputarrow]{} ++(0,0);
    %
    \draw[line width=0.8pt] (8.5,1.5) rectangle ++ (2,1);
    \draw (9.5,2) node[](){B};
    \draw (10.5,2) -- ++(1,0) to node[inputarrow]{} ++(0,0);
    %
    \draw[line width=0.8pt] (11.5,1.5) rectangle ++ (2,1);
    \draw (12.5,2) node[](){Endstufe};
    \draw (14.5,2) node [antenna]{};
    \draw (13.5,2) -- (14.5,2);
    \draw (14,2) circle (0.15cm);
    \draw (14,1.85) node[ground] (gnd2) {};
    %
    \draw (15.2,1.6) node[](){431,1\,MHz};
\end{circuitikz}
\end{document}
