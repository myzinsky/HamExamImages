% Author: Dr. Matthias Jung, DL9MJ
% Year: 2020
\documentclass[convert = false, border=5pt]{standalone}
\usepackage{fontspec}
\setmainfont{Roboto}
\usepackage[siunitx, straightvoltages, europeanresistors, european inductor]{circuitikzgit}
\usepackage{tikz}



\begin{document}

\begin{circuitikz}
    %\tikzstyle{help lines}=[blue!50];
    %\draw[style=help lines] (0,-2) grid (10,4);
    \draw (1.5,3) node [antenna]{}
        to [bandpass, >, label=1.Filter] ++(2,0)
        to [amp, box, >, label=HF] ++(2,0);
    \draw (6.5,3.82) node[] {Mischer};
    \draw (6.5,3) node[mixer, box] (mix) {} ++(1,0)
        to [bandpass, >, label=2.Filter] ++(2,0)
        to [short, -o] ++(0.5,0);
    \draw (10.35,3) node [] {RX};
    \draw (3.5,-0.1) node[spdt, rotate=-90] (schalter) {}; 
    \draw(schalter.in) node[] {};
    \draw(3.5,0.5) to [allpass,label=CO] ++(0,1);
    \draw(4,1) to (4.5,1) to [twoportsplit, >, t1={f}, t2={9f}] ++(2,0)
        to (6.5,1)
        to (mix.south);
    \draw (3,-1.5) node[ground] (gnd1) {}
    to [piezoelectric,/tikz/circuitikz/bipoles/length=0.7cm, >] ++(0,0.6) to (schalter.out 2);
    \draw (4,-1.5) node[ground] (gnd1) {}
    to [piezoelectric,/tikz/circuitikz/bipoles/length=0.7cm, >] ++(0,0.6) to (schalter.out 1);
    \draw (mix.south) node[inputarrow, rotate=90] {} to [short] ++(0,-1.5);
    \draw (mix.west) node[inputarrow] {} to [short] ++(-0.5,0);
    \draw (mix.east) to [short] ++(0.5,0);
    \draw (8.5,2.15) node[] {28...30\,MHz};
    \draw (7,1) node[] {$f_{OSZ}$};
    \draw (2.5,2.15) node[] {430...434\,MHz};
\end{circuitikz}
\end{document}
