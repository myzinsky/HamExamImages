% Author: Dr. Matthias Jung, DL9MJ
% Year: 2021
\documentclass[convert = false, border=5pt]{standalone}
\usepackage{fontspec}
\setmainfont{Roboto}
\usepackage[siunitx, straightvoltages, europeanresistors, european inductor]{circuitikzgit}
\usepackage{tikz}


\usepackage[siunitx, straightvoltages]{circuitikzgit}
\usepackage{amsmath}
\usepackage{unicode-math}
\setmathfont{Fira Math}
\setmathfont[range=up]{Roboto}
\setmathfont[range=it]{Roboto-Italic}
\setmathfont[range=\int]{Fira Math}
\usepackage[euler]{textgreek}
\usetikzlibrary{arrows, arrows.meta}

\usepackage{etoolbox}
\makeatletter

\pgfcircdeclarebipole{}
    {\ctikzvalof{bipoles/tline/height}}{tline}
    {\ctikzvalof{bipoles/tline/height}}
    {\ctikzvalof{bipoles/tline/width}}
{   
    %% First find distance from startpoint to endpoint
    \pgfpointdiff{\pgfpointanchor{\ctikzvalof{bipole/name}start}{center}}
                 {\pgfpointanchor{\ctikzvalof{bipole/name}end}{center}}
    \pgfmathparse{veclen(\the\pgf@x,\the\pgf@y)}
    %% The coordinate system has been changed so that the origin is at the midpoint and
    %% the line is along the x axis. So shift back by half the length of the line, and 
    %% make the cylinder of width roughly the length of the line, with a 40pt setback
    %% on each side.
    \pgftransformxshift{\pgfmathresult/2-30pt}
    \pgf@circ@res@left=\dimexpr-\pgfmathresult pt+40pt\relax
    %% Here is the original function, copied directly from the source of circuittikz, 
    %% down to next %%
    \pgf@circ@res@step=.2\pgf@circ@res@right % half x axis
    \pgfsetlinewidth{\pgfkeysvalueof{/tikz/circuitikz/bipoles/thickness}\pgfstartlinewidth}
    \pgfpathellipse{\pgfpoint{\pgf@circ@res@right-\pgf@circ@res@step}{0}}
                   {\pgfpoint{\pgf@circ@res@step}{0}}
                   {\pgfpoint{0}{-\pgf@circ@res@up}}
    \pgfpathmoveto{\pgfpoint{\pgf@circ@res@right-\pgf@circ@res@step}{\pgf@circ@res@up}}
    \pgfpathlineto{\pgfpoint{\pgf@circ@res@left+\pgf@circ@res@step}{\pgf@circ@res@up}}
    \pgfpatharc{-90}{90}{-\pgf@circ@res@step and -\pgf@circ@res@up}
    \pgfpathlineto{\pgfpoint{\pgf@circ@res@right-\pgf@circ@res@step}{\pgf@circ@res@down}}
    %% I have to fill the figure to block out the original line
    \pgfsetfillcolor{white}
    \pgfusepath{draw,fill}
    %% Redraw part of the line that gets blocked by the cylinder by mistake
    \pgfpathmoveto{\pgfpoint{\pgf@circ@res@right-2*\pgf@circ@res@step}{0pt}}
    \pgfpathlineto{\pgfpoint{\pgf@circ@res@right}{0pt}}
    \pgfusepath{draw}
}

\begin{document}

\begin{circuitikz}
    %\tikzstyle{help lines}=[blue!50];
    %\draw[style=help lines] (-2,-2) grid (18,8);

    \draw (-1.2,2) node[]{...};
    \draw (0.5,2) node[tlinestub, rotate=180](HF){}
        to [twoport, t=MWS] ++ (1,0)
        to [TL] ++(3.0,0)
        to [short, -*] ++(0,0) coordinate(c1)
        to [short] ++(0.5,0); 
    \draw (5,2) to [american inductor, inductors/coils=2 , inductors/width=0.45]
        ++(0,-1);
    \draw (6,1) to [american inductor, inductors/coils=14, inductors/width=3.25]
        ++(0,5) coordinate(feed);
    \draw[thick]  (feed) -- ++ (8,0);
    \draw[dashed] (5.5,1.25) -- ++ (0,4.5);

    \draw(5,1) to [short,-*] ++(-0.5,0) coordinate(c2)
        to [short] ++(-0.7,0)
        to [short,-*] ++(0,0.8);

    \draw(6,1) to [short,-*] ++(-1,0);

    \draw(c1) to [C, capacitors/scale=0.5] (c2);

    % Kurze Striche:
    \draw ( 6.0,6.3) -- ++ (0,0.4);
    \draw (14.0,6.3) -- ++ (0,0.4);

    \draw (1.9,2.4) -- ++ (0,0.4);
    \draw (3.8,2.4) -- ++ (0,0.4);

    % Pfeile:
    \draw[>=triangle 60, <-] ( 6.0,6.5) -- ++(3.5,0);
    \draw[>=triangle 60, ->] (10.5,6.5) -- ++(3.5,0);
    \draw (10,6.5) node[]{$\approx\dfrac{\lambda}{2}$};

    \draw[>=triangle 60, <-] ( 1.9,2.6) -- ++(0.35,0);
    \draw[>=triangle 60, ->] ( 3.4,2.6) -- ++(0.35,0);
    \draw (2.85,2.6) node[]{\small{$0{,}05\cdot\lambda$}};

    \draw (5,4) node[]{\small{ü=$\dfrac{1}{7}$}};

\end{circuitikz}
\end{document}

